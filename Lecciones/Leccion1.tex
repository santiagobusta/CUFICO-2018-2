\documentclass[10.5pt]{article}

% Spanish characters
\usepackage[utf8]{inputenc}
\usepackage[T1]{fontenc}
% French display
\usepackage[english,spanish]{babel}

\usepackage{lastpage}
%Esto me permite usar el comando "\pageref{LastPage}" en el footer.
\renewcommand{\baselinestretch}{1.6}
% Esto controla el interlineado o espaciado!!!
\usepackage{color}
%\newcommand{\red}[1]{{\color{red} #1}}
\newcommand{\red}[1]{{\color{black} #1}}

%Esto me permite poner hipervínculos:
%\usepackage[pdftex,
%       colorlinks=true,
%       urlcolor=blue,       % \href{...}{...} external (URL)
%       filecolor=green,     % \href{...} local file
%       linkcolor=black,       % \ref{...} and \pageref{...}
%       pdftitle={Papers by AUTHOR},
%       pdfauthor={Your Name},
%       pdfsubject={Just a test},
%       pdfkeywords={test testing testable},
%%       pagebackref,%Esto parece que pone un numerito al lado de la referencia (en la seccion de bibliografia), donde se puede clicar y te lleva al lugar del texto donde se le cita.
%       pdfpagemode=None,
%       bookmarksopen=true]{hyperref}


%The following packages are relics, but I don't want to remove them just in case:
\usepackage{amsmath}
\usepackage{array}
\usepackage{latexsym}
\usepackage{amsfonts}
\usepackage{amsthm}
\usepackage{cite}
\usepackage{setspace}
\usepackage{amssymb}
\usepackage{hyperref}

\usepackage{multicol}
\usepackage{color}
%\usepackage{minipage}

\usepackage{graphicx} % Required for including images
\graphicspath{{figures/}} % Location of the graphics files
\usepackage[font=small,labelfont=bf]{caption} % Required for specifying captions to tables and figures

%The defaults margins are huge, so I'll customize it:
\oddsidemargin  -0.0 in
\textwidth 6.5 in
\textheight 8.7 in
\addtolength{\voffset}{-1cm}

%%%%%%%%%%%%%%%%%%%%%%%% HEADER AND FOOTER %%%%%%%%%%%%%%%%%%%%
\usepackage{fancyhdr}
\pagestyle{fancy}

\fancyhead[L]{Lecci\'{o}n 1}
%\fancyhead[L]{CNRS Competition 01-04}
\fancyhead[R]{Jos\'{e} David Ruiz \'{A}lvarez}
\fancyhead[C]{}
\fancyfoot[C]{\thepage /\pageref{LastPage}}

\newlength\FHoffset
\setlength\FHoffset{0cm}

\addtolength\headwidth{2\FHoffset}
\fancyheadoffset{\FHoffset}

\newlength\FHleft
\newlength\FHright

\setlength\FHleft{1cm}
\setlength\FHright{1cm}

\thispagestyle{empty}
%%%%%%%%%%%%%%%%%%%%%%%% HEADER AND FOOTER %%%%%%%%%%%%%%%%%%%%



\begin{document}

%\begin{center}
\noindent
\begin{minipage}[b]{0.75\linewidth}
{\LARGE\bf Lecci\'{o}n 1}\\ %[1mm]
%\end{center}
%{\Large\bf \emph{}}\\ %[3mm]
%{\Large\bf \emph{connections between LHC and neutrino experiments}}
\large{Jos\'{e} David Ruiz \'{A}lvarez} \\
\small{\href{mailto:jose.ruiz@cern.ch}{jose.ruiz@cern.ch}} \\ %[3mm]
%\normalsize{Plaza código: 2017010307, Área: Física de fenomenología de altas energías} \\%[3mm]
\normalsize{Instituto de Física, Facultad de Ciencias Exactas y Naturales} \\%[3mm]
\normalsize{\bf Universidad de Antioquia} \\[8mm]
\today %\\[4mm]
\end{minipage}%
%\end{center}
%\begin{minipage}[b]{0.25\linewidth}
%\centering{\includegraphics[width=4cm]{figures/CMS.png}}\\
%%%%%\includegraphics[width=15cm]{figures/UniandesColombia.jpg}\\
%\end{minipage}

%\begin{center}
%{\bf Palabras clave:} CERN, LHC, CMS, Materia Oscura
%\end{center}

%\doublespacing

\section{Contenido}

Dos grandes vertientes en t\'{e}rminos de problemas f\'{i}sicos: 
\begin{itemize}
\item Análisis estadístico de datos
\item Solución numérica de ecuaciones diferenciales
\end{itemize}

En análisis de datos tendremos como objetivos específicos:
\begin{itemize}
\item Definición de un conjunto de datos 
\item Preparación de un conjunto de datos para hacer una medida
\item Visualización de un conjunto de datos
\item Distribuciones de probabilidad
\item Fit de un conjunto de datos a una función
\item Medición de un observable
\item Error en la medida
\end{itemize}

Con respecto a la solución numérica de ecuaciones diferenciales:
\begin{itemize}
\item Conceptos básicos de análisis numérico
\item Problemas de valor inicial para ecuaciones diferenciales ordinarias
\item Método de Euler
\item Método de Runge-Kutta
\item Errores de redondeo
\item Consistencia, convergencia y estabilidad
\end{itemize}

\section{Evaluación}

\begin{itemize}
\item 20\% seguimiento y tareas: Problemas cortos y ejercicios de programación.
\item 30\% proyecto: Problema físico a resolver en grupos.
\item 50\% dos parciales de igual valor. 
\end{itemize}

\section{Otros}

\begin{itemize}
\item Lenguajes de programación: C(++) y python
\item Paquetes: ROOT, numpy, matplotlib, entre otros.
\end{itemize}


%\begin{center}
%\includegraphics[width=0.4\linewidth]{DM_detection.jpg}
%\captionof{figure}{Diagrama ilustrativo donde se muestran los 3 procesos principales que se utilizan para la detección de materia oscura. En el sentido de la línea verde se muestra el proceso de producción de materia oscura en colisionadores, la línea roja corresponde a la detección directa y la línea azul a la detección indirecta.}
%\label{fig:DMdetection}
%\end{center}



%\singlespacing
%\begin{thebibliography}{99}
%
%\end{thebibliography}

\end{document}

%%% Local Variables:
%%%   mode: latex
%%%   mode: flyspell
%%%   ispell-local-dictionary: "spanish"
%%% End:
